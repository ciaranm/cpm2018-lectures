% vim: set spell spelllang=en tw=100 et sw=4 sts=4 foldmethod=marker foldmarker={{{,}}} :

\documentclass{beamer}

\usepackage{tikz}
\usepackage{xcolor}
\usepackage{complexity}
\usepackage{hyperref}
\usepackage{microtype}
\usepackage{amsmath}                   % \operatorname
\usepackage{amsfonts}                  % \mathcal
\usepackage{amssymb}                   % \nexists
\usepackage{gnuplot-lua-tikz}          % graphs
\usepackage[vlined]{algorithm2e} % algorithms
\usepackage{centernot}
\usepackage{mathtools}
\usepackage{listings}

\usetikzlibrary{shapes, arrows, shadows, calc, positioning, fit}
\usetikzlibrary{decorations.pathreplacing, decorations.pathmorphing, shapes.misc}
\usetikzlibrary{tikzmark}

\definecolor{uofguniversityblue}{rgb}{0, 0.219608, 0.396078}

\definecolor{uofgheather}{rgb}{0.356863, 0.32549, 0.490196}
\definecolor{uofgaquamarine}{rgb}{0.603922, 0.72549, 0.678431}
\definecolor{uofgslate}{rgb}{0.309804, 0.34902, 0.380392}
\definecolor{uofgrose}{rgb}{0.823529, 0.470588, 0.709804}
\definecolor{uofgmocha}{rgb}{0.709804, 0.564706, 0.47451}
\definecolor{uofgsandstone}{rgb}{0.321569, 0.278431, 0.231373}
\definecolor{uofgforest}{rgb}{0, 0.2, 0.129412}
\definecolor{uofglawn}{rgb}{0.517647, 0.741176, 0}
\definecolor{uofgcobalt}{rgb}{0, 0.615686, 0.92549}
\definecolor{uofgturquoise}{rgb}{0, 0.709804, 0.819608}
\definecolor{uofgsunshine}{rgb}{1.0, 0.862745, 0.211765}
\definecolor{uofgpumpkin}{rgb}{1.0, 0.72549, 0.282353}
\definecolor{uofgthistle}{rgb}{0.584314, 0.070588, 0.447059}
\definecolor{uofgrust}{rgb}{0.603922, 0.227451, 0.023529}
\definecolor{uofgburgundy}{rgb}{0.490196, 0.133333, 0.223529}
\definecolor{uofgpillarbox}{rgb}{0.701961, 0.047059, 0}
\definecolor{uofglavendar}{rgb}{0.356863, 0.301961, 0.580392}

\tikzset{vertex/.style={draw, circle, inner sep=0pt, minimum size=0.5cm, font=\small\bfseries}}
\tikzset{notvertex/.style={vertex, color=white, text=black}}
\tikzset{plainvertex/.style={vertex}}
\tikzset{vertexc1/.style={vertex, fill=uofgburgundy, text=white}}
\tikzset{vertexc2/.style={vertex, fill=uofgsandstone, text=white}}
\tikzset{vertexc3/.style={vertex, fill=uofgforest, text=white}}
\tikzset{vertexc4/.style={vertex, fill=uofgheather, text=white}}
\tikzset{edge/.style={color=black!50!white}}
\tikzset{bedge/.style={ultra thick}}
\tikzset{edged/.style={color=screengrey, dashed}}
\tikzset{edgel3/.style={color=uofgrose, ultra thick}}

% {{{ theme things
\useoutertheme[footline=authortitle]{miniframes}
\useinnertheme{rectangles}

\setbeamerfont{block title}{size={}}
\setbeamerfont{title}{size=\large,series=\bfseries}
\setbeamerfont{section title}{size=\large,series=\mdseries}
\setbeamerfont{author}{size=\normalsize,series=\mdseries}
\setbeamercolor*{structure}{fg=uofguniversityblue}
\setbeamercolor*{palette primary}{use=structure,fg=black,bg=white}
\setbeamercolor*{palette secondary}{use=structure,fg=white,bg=uofgcobalt}
\setbeamercolor*{palette tertiary}{use=structure,fg=white,bg=uofguniversityblue}
\setbeamercolor*{palette quaternary}{fg=white,bg=black}

\setbeamercolor*{titlelike}{parent=palette primary}

\beamertemplatenavigationsymbolsempty

\setbeamertemplate{title page}
{
    \begin{tikzpicture}[remember picture, overlay]
        \node at (current page.north west) {
            \begin{tikzpicture}[remember picture, overlay]
                \fill [fill=uofguniversityblue, anchor=north west] (0, 0) rectangle (\paperwidth, -2.6cm);
            \end{tikzpicture}
        };

        \node (logo) [anchor=north east, shift={(-0.6cm,-0.6cm)}] at (current page.north east) {
            \includegraphics[keepaspectratio=true,scale=0.7]{UoG_keyline.pdf}
        };

        \node [anchor=west, xshift=0.2cm] at (current page.west |- logo.west) {
            \begin{minipage}{0.65\paperwidth}\raggedright
                {\usebeamerfont{title}\usebeamercolor[white]{}\inserttitle}\\[0.1cm]
                {\usebeamerfont{author}\usebeamercolor[white]{}\insertauthor}
            \end{minipage}
        };
    \end{tikzpicture}
}

\setbeamertemplate{section page}
{
    \begin{centering}
        \begin{beamercolorbox}[sep=12pt,center]{part title}
            \usebeamerfont{section title}\insertsection\par
        \end{beamercolorbox}
    \end{centering}
}

\newcommand{\frameofframes}{/}
\newcommand{\setframeofframes}[1]{\renewcommand{\frameofframes}{#1}}

\makeatletter
\setbeamertemplate{footline}
{%
    \begin{beamercolorbox}[colsep=1.5pt]{upper separation line foot}
    \end{beamercolorbox}
    \begin{beamercolorbox}[ht=2.5ex,dp=1.125ex,%
        leftskip=.3cm,rightskip=.3cm plus1fil]{author in head/foot}%
        \leavevmode{\usebeamerfont{author in head/foot}\insertshortauthor}%
        \hfill%
        {\usebeamerfont{institute in head/foot}\usebeamercolor[fg]{institute in head/foot}\insertshortinstitute}%
    \end{beamercolorbox}%
    \begin{beamercolorbox}[ht=2.5ex,dp=1.125ex,%
        leftskip=.3cm,rightskip=.3cm plus1fil]{title in head/foot}%
        {\usebeamerfont{title in head/foot}\insertshorttitle}%
        \hfill%
        {\usebeamerfont{frame number}\usebeamercolor[fg]{frame number}\insertframenumber~\frameofframes~\inserttotalframenumber}
    \end{beamercolorbox}%
    \begin{beamercolorbox}[colsep=1.5pt]{lower separation line foot}
    \end{beamercolorbox}
}

% }}}

\title{Parallel Constraint Programming}
\author[Ciaran McCreesh and Patrick Prosser]{\textbf{Ciaran McCreesh} and Patrick Prosser}

\begin{document}

{
    \usebackgroundtemplate{
        \tikz[overlay, remember picture]
        \node[at=(current page.south), anchor=south, inner sep=0pt]{\includegraphics[keepaspectratio=true, width=\paperwidth]{background3.jpg}};
    }
    \begin{frame}[plain,noframenumbering]
        \titlepage
    \end{frame}
}

\begin{frame}{Parallel In Theory}
\end{frame}

\begin{frame}{Parallel Optimisation?}
\end{frame}

\begin{frame}{Parallel Consistency?}
\end{frame}

\begin{frame}{Simple Parallel Search}
\end{frame}

\begin{frame}{Random Work Stealing}
\end{frame}

\begin{frame}{Embarrassingly Parallel Search}
\end{frame}

\begin{frame}{Parallel Discrepancy Search}
\end{frame}

\begin{frame}{Confidence-Based Work Stealing}
\end{frame}

\begin{frame}{Parallel Portfolios?}
\end{frame}

\begin{frame}{This is Not The Exam Question}

    \scriptsize

    A constraint model takes 10 seconds to solve using one processor. Suppose 80\% of that time is
    spent doing propagation. What is the best possible speedup that could be obtained if 4
    processors are used to do parallel propagation, and the rest of the program remains
    unchanged? What about if we had an unlimited number of processors?
    \\[0.5cm]

    What about if we used the four processors for a portfolio of different solvers? \\[0.5cm]

    What is \emph{balance}, and why is it a problem if we try to parallelise a tree-search by
    creating $n$ sub-trees for $n$ processors? Suggest two potential remedies. \\[0.5cm]

    Suppose we are solving a decision problem which has a sequential part taking
    one second to run, and a parallelisable part which takes twenty seconds to run on one
    processor. What is the best possible runtime we might see when using ten processors to
    solve this problem with a parallel tree-search, if the instance is satisfiable? What if
    it is unsatisfiable? \\[0.5cm]

    Design a parallel constraint programming approach that works perfectly.

\end{frame}

\begin{frame}[plain,noframenumbering]
    \begin{tikzpicture}[remember picture, overlay]
        \node at (current page.north west) {
            \begin{tikzpicture}[remember picture, overlay]
                \fill [fill=uofguniversityblue, anchor=north west] (0, 0) rectangle (\paperwidth, -1.7cm);
            \end{tikzpicture}
        };

        \node (logo) [anchor=north east, shift={(-0.3cm,-0.2cm)}] at (current page.north east) {
            \includegraphics[keepaspectratio=true,scale=0.55]{UoG_keyline.pdf}
        };
    \end{tikzpicture}
\end{frame}

\end{document}


